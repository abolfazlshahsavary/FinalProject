\documentclass[12pt]{article}

\usepackage[utf8]{inputenc}
\usepackage{geometry}
\geometry{a4paper, margin=1in}
\usepackage{graphicx}
\usepackage{hyperref}
\usepackage{fancyhdr}

\setlength{\headheight}{15pt}
\pagestyle{fancy}
\fancyhf{}
\rhead{Computer Workshop Course}
\lhead{Final Assignment}
\rfoot{Page \thepage}

\title{Final project}
\author{abolfazl shahsavari}
\date{January 2024}

\begin{document}

\maketitle
\newpage
\tableofcontents
\newpage

\section{repository link}
\paragraph{\url{https://github.com/abolfazlshahsavary/FinalProject}}
\section{set up the repository}
\paragraph{To set up a repository, first of all,\\ we must have a GitHub account. After creating a GitHub account or \\logging into the account, we perform the following steps:}

\begin{itemize}
    \item Go to the repository section in your account
    \item Click on New Repository
    \item Select the repository owner and its name
    \item Enter the explation in the description field
    \item Choose whether your repository is public or private
    \item If you want to add a README to your repository and edit it later
    \item you can add licence to tell others the ablity of your code and what they can do with your code
    \item finaly click on creat repository 
\end{itemize}
\section{vim advance feature}
\begin{itemize}
    \item Vim Macros: Macros in Vim allow you to record a series of keystrokes and \\then play them back repeatedly. This feature can save you a lot of time when\\ performing repetitive tasks.  
    \item Vim Foldings: Foldings in Vim allow you to collapse and expand sections of code, making it easier to navigate and work with large files
    \item  Vim Registers: Registers in Vim allow you to store text snippets that you can easily paste into your document

\end{itemize}
\section{Memory profiling}
\paragraph{Memory leaks occur when a program allocates memory but fails to release it when it's no longer needed, leading to the waste of system resources and potentially causing performance issues or crashes. In other words, memory leaks result in the persistent use of memory that should have been freed up.

the most important reson for this problem is:\\

Forgetting to free dynamically allocated memory: In languages like C and C++, memory is allocated using functions like malloc or new, and it's the responsibility of the programmer to free the memory using functions like free\(\) or delete when it's no longer required. If the programmer forgets to free the memory, it will continue to occupy the allocated space, leading to a memory leak.\\ 
Valgrind is a powerful tool used for debugging and profiling Linux-based systems. Its primary purpose is to help developers identify and fix memory leaks, segmentation faults, and other memory-related issues in their programs.

Valgrind works by instrumenting the executable code of the program being debugged, which allows it to monitor the allocation and deallocation of memory, as well as other low-level system calls. It then generates detailed reports on the program's behavior, highlighting any potential issues and providing suggestions for improvement.

When memory leaks occur, Valgrind can help by identifying the specific lines of code where the memory is being allocated but not freed. This information can then be used to locate the source of the leak and develop a solution to prevent it from happening in the future.

Valgrind also provides detailed information on memory usage, including the amount of memory being allocated and deallocated, as well as any potential memory corruption issues that may arise due to incorrect pointer usage or buffer overflows/underflows in the program's codebase.. This information can help developers optimize their programs for memory usage and improve overall performance by reducing the amount of unnecessary memory allocation and deallocation that can lead to performance bottlenecks and slowdowns in large-scale applications}




\end{document}