\documentclass[12pt]{article}

\usepackage[utf8]{inputenc}
\usepackage{geometry}
\geometry{a4paper, margin=1in}
\usepackage{graphicx}
\usepackage{hyperref}
\usepackage{fancyhdr}

\setlength{\headheight}{15pt}
\pagestyle{fancy}
\fancyhf{}
\rhead{Computer Workshop Course}
\lhead{Final Assignment}
\rfoot{Page \thepage}

\title{Final project}
\author{abolfazl shahsavari}
\date{January 2024}

\begin{document}

\maketitle
\newpage
\tableofcontents
\newpage

\section{repository link}
\paragraph{\url{https://github.com/abolfazlshahsavary/FinalProject}}
\section{set up the repository}
\paragraph{To set up a repository, first of all,// we must have a GitHub account. After creating a GitHub account or //logging into the account, we perform the following steps:}

\begin{itemize}
    \item Go to the repository section in your account
    \item Click on New Repository
    \item Select the repository owner and its name
    \item Enter the explation in the description field
    \item Choose whether your repository is public or private
    \item If you want to add a README to your repository and edit it later
    \item you can add licence to tell others the ablity of your code and what they can do with your code
    \item finaly click on creat repository 
\end{itemize}

\end{document}