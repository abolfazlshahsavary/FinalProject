\documentclass[12pt]{article}

\usepackage[utf8]{inputenc}
\usepackage{geometry}
\geometry{a4paper, margin=1in}
\usepackage{graphicx}
\usepackage{hyperref}
\usepackage{fancyhdr}

\setlength{\headheight}{15pt}
\pagestyle{fancy}
\fancyhf{}
\rhead{Computer Workshop Course}
\lhead{Final Assignment}
\rfoot{Page \thepage}

\title{
    \vspace{2in}
    \textbf{Final Assignment:}\\
    \textbf{Integration of Tools and Practices}\\
    \large Iran University of Science and Technology\\
    \large Department of Computer Engineering\\
    \vspace{2in}
}

\author{
    \vspace{0.5in}
    Dr.\ MalekiMajd\\
    Computer Workshop 02-03\\
    \vspace{0.5in}
}

\date{Due Date: 7 Bahman}
\begin{document}

\begin{titlepage}
    \maketitle
    \thispagestyle{empty}
\end{titlepage}

\newpage

\section*{Introduction}
This final assignment is designed to encapsulate the core skills and knowledge you have acquired throughout the course. It will require you to apply all the tools and practices covered, including Git, LaTeX, and more. This assignment must be submitted as a LaTeX document managed under version control using Git, with the repository hosted on GitHub and the document compiled using GitHub Actions.

\section*{Instructions}
\begin{enumerate}
    \item Create a new GitHub repository and clone it to your local machine.
    \item Write your assignment document in LaTeX, commit your changes regularly, and push them to your GitHub repository.
    \item Set up GitHub Actions to compile your LaTeX document to a PDF upon every tag in the repository.
    \item The document should include sections for each category of the questions.
\end{enumerate}

\subsection*{More on GitHub Actions}
We have already experimented with GitHub actions in the al-folio template before. In this sections, you will be using GitHub actions to compile the LaTeX file you write and upload it on the "Releases section" of your repository.

In order to do this, here are some links that help you achieve this:
\begin{itemize}
    \item \url{https://mrturkmen.com/posts/build-release-latex/} \\
        This link will overall help you to understand what is going on in the Action.
    \item \url{} \\
        This is repository created especially for this assignment, it has the GitHub action already setup and running. feel free to use the workflow \texttt{.yml}
        for your own usage.
\end{itemize}

\paragraph{NOTE:} The Action is triggered on every new \textbf{tag}. Learn about tags by searching and make sure to tag the commit you want to compile to PDF using
GitHub Actions.

\subsection*{LaTeX Document Guidelines}
You must create a structured LaTeX document including a title page, sections for each task, and a page for table of contents.


\pagebreak


\section*{Assignments}
Here, you will find the questions/problems you have to answer in your LaTeX file.
\paragraph{NOTE:} Make sure to write them in your LaTeX file and make commits for each section you write.

\section{Git and GitHub}
\subsection{Repository Initialization and Commits}
Write about how you set up the repository for this assignment. Explain every step in detail.

\subsection{GitHub Actions for LaTeX Compilation}
Provide a walkthrough of setting up GitHub Actions to automatically compile your LaTeX document and any challenges you encountered.


\section{Exploration Tasks}
\subsection{Vim Advanced Features}
Explore and document 3 advanced features of Vim that were not covered in class.

\subsection{Memory profiling}
This semester, you got to know about dynamic memory allocation in C in your Programming Fundamentals class.

\subsubsection{Memory Leak}
In short, explain what memory leaks are and how they might happen in your program.

\subsubsection{Memory profilers}
Read about a tool called \textit{Valgrind} and write about their purpose and how it helps when memory leaks happen.

\subsection{GNU/Linux Bash Scripting}
In this section, you will get to know some handy bash utilities.

\subsubsection{fzf}
Read about a handy CLI tool called \textit{fzf} and answer the following questions:

\begin{itemize}
    \item What is fuzzy searching? Give a short description.
    \item Install fzf on your machine and give a description of what the following command does:\\
        \texttt{ls | fzf}
\end{itemize}

\subsubsection{Using fzf to find your favorite PDF}
You might have came across moments when you want to open up a certain PDF when studying for your final exams but finding the directory of that
PDF is a very gruesome and tiring process. In this section, we will be using \textit{fzf} to find our PDF in seconds! We will be going step by step
on how to find your file and use fzf to select it.

\begin{enumerate}
    \item We first need to list the directory of all the files with the extension \texttt{.PDF}. Write a command to list the directory of all the files with the extension \texttt{.PDF} \\
        \textbf{HINT:} use the command \texttt{fd} for this purpose.
    \item Now we have to select the PDF we want using \textit{fzf}. Write a command to use \textit{fzf} to select a PDF from the data we gathered above.
\end{enumerate}

\subsubsection{Opening the file using Zathura}
Now that have selected which PDF we want to open, we can use a very minimalistic program called \textit{Zathura} to open it. Write a command that uses the
commands above to open the file using Zathura.
\paragraph{HINT:} This is how Zathura opens files: \texttt{zathura /path/to/file}
\paragraph{HINT:} Learn what the syntax \texttt{zathura \$(\textit{command})} does and use to open the PDF we have selected.

\section{Git and FOSS}
\subsubsection{README.md}
Make sure to include a basic README.md file in your GitHub repository that describes the aim of this repository and its purpose.

\subsection{Issues}
Create a sample issue in the repository below and attach its screenshot in your LaTeX document:

\url{}

\pagebreak

\section*{Submission Guidelines}
Your final submission should include:
\begin{itemize}
    \item A link to your GitHub repository.
    \item The compiled PDF of your LaTeX document.
    \item All source files used in the creation of the document and scripts.
\end{itemize}

\section*{Evaluation Criteria}
You will be evaluated on the following:
\begin{itemize}
    \item The completeness and quality of your LaTeX document.
    \item The use of Git and GitHub, including commit history and usage of GitHub Actions.
    \item The depth of exploration and learning in each of the tasks.

\item Your ability to follow the submission guidelines accurately.
\end{itemize}

Good luck!

\end{document}
